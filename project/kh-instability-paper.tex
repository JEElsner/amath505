\documentclass{article}

\usepackage{fullpage}
\usepackage{amsmath}
\usepackage{bm}
\usepackage{esdiff}
\usepackage{cite}
\usepackage{hyperref}
\usepackage{graphicx}

\renewcommand{\vec}[1]{\bm{\mathrm{#1}}}

\graphicspath{{./res/}}

\title{Kelvin-Helmoltz Instability}
\author{Jonathan Elsner}
\date{December 11\textsuperscript{th}, 2024}

\begin{document}

\maketitle

\section{Abstract}

% Maybe look at MIT guide to abstract writing?
Kelvin-Helmholtz (KH) instability is a process that occurs at the interface
between two fluids of two differing densities, creating vortex waves. KH
Instability is a commonly occurring process in nature: it is one of the driving
forces of mixing in the ocean \cite{woods-1968}, causes clear-air turbulence on
flights and can sometimes be seen in cloud formations \cite{ludlam-1967}. It is
also thought to be a factor in heating the corona of the Sun
\cite{nasa-solar-surfer}.

\begin{figure}[h]
    \centering
    \includegraphics[width=4in]{kh-instability-clouds-2.jpg}
    \caption{Rare fluctus clouds are caused by KH Instability \cite{ludlam-1967}. Photo: \cite{fluctus-clouds}.}
    \label{img:fluctus-clouds}
\end{figure}

The key characteristic of KH instability is the distance between the waves at
the interface. This changes over time and is dependent on both the difference in
density between the fluids and the flow rate of each fluid \cite{kundu}.

We follow a procedure outlined in \cite{kh-instability-demo} to recreate this
phenomenon in the lab to analyze the results and determine the relationship
between fluid speed, fluid density, and instability wavelength. We also analyze
the evolution of the waves over time.


\section{Theory}

KH instability occurs when the energy from velocity overcomes forces stabilizing
the stratification between the fluids. Velocity creates a shear force that acts
on the interface, amplifying small disturbances on the interface. Small
irregularities in the velocity create pressure differences due to Bernoulli's
law: high velocities correspond with low pressure, and low velocity creates high
pressure. These pressure differences amplify the perturbations to create the
phenomenon. Waves develop because the differing velocities give rise to
vorticity in the fluid.

We can characterize the likelihood of instability with the Richardson Number,
which is the ratio of stabilizing forces (in our case, gravity) to shear forces
due to velocity. From Kundu \cite{kundu}, it is defined as:

\[ \text{Ri} = \frac{N^2}{(dU/dz)^2} \quad\text{with}\quad N^2 = \frac{-g}{\rho_0} \diff{\rho}{z} \]

where \(\rho_0\) is a reference pressure. This formulation accounts for the
continuous case where density and velocity change continuously; simplifying and
formulating for the discrete case where velocity and density change suddenly
across the interface (i.e. \(dU/dz = \Delta U / \Delta z\) and the same for
\(\rho\)):

\[ \text{Ri} = \frac{-g}{\rho_0} \frac{\Delta \rho / \Delta z}{(\Delta U / \Delta z)^2} = \frac{-g}{\rho_0} \frac{\Delta \rho \Delta z}{\Delta U}\]

The Richardson number is small when instability is present. In fact, \(\text{Ri}
< 1/4\) is a necessary condition for instability. In our experiment, we should
expect to find Richardson Numbers less than this when KH instability occurs.

\section{Procedure}

To carry out the experiment, we followed the following rough procedure, adopted
from \cite{kh-instability-demo}:

\begin{enumerate}
    \item Fill tank halfway with fresh water
    \item Mix salt for desired salinity. We calculated how much salt we wanted
    for a desired concentration, mixed that together, then verified the salinity
    with a handheld refractometer.
    \item Slowly fill the tank to full with the denser salt water.
    \item Return tank to level and allow the interface to stabilize
    \item Swiftly tilt tank to develop the instability.
\end{enumerate}

\begin{figure}[h!]
    \centering
    \includegraphics[width=5in]{filling-with-salt-water-4.jpg}
    \caption{Carefully returning the tank to level after filling.}
    \label{img:leveling}
\end{figure}

The denser fluid was dyed darkly to create a stark visual boundary between the
two fluids. We placed a light sheet behind the tank to form a uniform background
and consistently illuminate the fluid. We recorded video of the flow both on a
small scale in slow motion and at a wide angle capturing the entire tank.

We attempted to keep the rotation angle of the tank consistent by stopping the
rotation of the tank on a bucket placed under one end of the apparatus.

We used a tank that is 2.834 meters long, 0.1905 meters tall, and 0.054 meters
wide.

\begin{figure}[h!]
    \centering
    \includegraphics[width=5in]{15pct-screenshot.png}
    \caption{Close up of the waves formed in the trial with 15\% salinity.}
    \label{img:15pct-waves}
\end{figure}

\section{Results and Analysis}

We calculated the density of the saltwater using GSW-Python, an implementation
of the Thermondynamic Equation of Seawater, assuming that the temperature was
\(23^\circ\) C.

Velocity data was not readily available from the videos of each run, so we
derived the velocity of each half of the tank using data on the height of the
interface. We used \href{https://physlets.org/tracker/}{Tracker} to extract the
motion of key components of the videos. We determined wave number by measuring
the distance between crests of the visible waves.

Fluid velocity was assumed to be equal and opposite for both sides of the fluid,
and determined by estimating the fluid flow. We measured the height of the
interface at both ends of the channel to calculate the volume of each liquid in
a defined space. To measure the height, we tracked the position of points on the
interface and the angle of the channel. We then transformed the positions so
that they were fixed relative to the tank; that is, we observed the positions
from a reference frame tracking the rotation of the channel.
\begin{figure}[h]
    \centering
    \includegraphics[width=5in]{tank-schematic.png}
    \caption{Schematic of the Tank. Coordinates were rotated by $-\theta$ so that the coordinate axes are parallel and perpendicular to the bottom of the channel.}
    \label{img:tank-schematic}
\end{figure}
Tracking the height over time gives us the flow rate of the liquid, which we can
divide by the cross section of the channel to find the fluid velocity. The
velocity varies over time, so we chose to average the velocity during the time
after the channel was tilted, but before the instability started. We then used
this data to compute the Richardson Number, assuming that \(\Delta z\) is the
height of the tank.

\begin{table}[h!]
    \centering
    \begin{tabular}{|c|c|c|c|c|c|}
    \hline
    \textbf{Salinity (\%)} & $\bm{\Delta \rho}$ \textbf{(kg/m\textsuperscript{3})} & $\bm{\Delta U}$ \textbf{(m/s)}& \textbf{Ri} & $\bm{k}$ \textbf{(1/m)}& $\bm{\lambda}$ \textbf{(m)}\\ \hline
    3.5 & 0.636796 & 0.07 & 0.243153 & 43.744532 & 0.022860 \\
    9.5 & 1.089990 & 0.10 & 0.203845 & 56.242970 & 0.017780 \\
    15.0 & 1.504392 & 0.12 & 0.195297 & 65.616798 & 0.015240 \\
    \hline
    \end{tabular}
    \caption{Summary of collected and derived data from the lab. \(\Delta \rho\) is the difference in density between the fresh water and the salt water. \(\Delta U\) is the difference in velocity between the water in the top and bottom half of the tank.}
    \label{tab:data}
\end{table}

In Figure \ref{graph:Ri-vs-k} we plot the relationship between Ri and \(k\). We
observe that \(\text{Ri} < 1/4\) for all trials, and that as the Richardson
Number increases, the observed wave number decreases non-linearly.

\begin{figure}[h!]
    \centering
    \includegraphics[width=5in]{RivsK.png}
    \caption{Wave number $k$ as a function of Ri.}
    \label{graph:Ri-vs-k}
\end{figure}

\section{Future Work}

Several aspects of this experiment proved challenging. While it was relatively
easy (albeit time consuming) to create the phenomenon in a lab, collecting
useful data on the instability was harder. As a result, much post-processing had
to be done to produce useable data. Better planning and experimental design
would have mitigated these issues. In particular, we should have had a clear
idea of the independent and dependent variables at play before running the
experiment, in order collect relevant data for these variables. A clear,
unobstructed video recording of the entire tank is recommended in any future
research. Further, consistency of parameters between trials is important. This
includes camera placement, degree of tank tilt, and speed of tank tilt.
Maintaining solid control variables will ensure that multiple trials can be
easily compared.

\section{Acknowledgements}

We would like to thank the members of our group: Jesse Akes, Anna Dodson, and
Maddy Kovaleski. They were instrumental in making this project happen; it truly
was a team effort. Beyond the project, Jesse and Maddy provided immense support
this quarter, both academically and socially. We would also like to thank Georgy
Manucharyan and Noah Rosenberg for creating a welcoming, low-pressure
environment in class and teaching us a lot.  

\newpage
\bibliography{bibliography}{}
\bibliographystyle{plain}

\end{document}