\documentclass{article}

\usepackage{fullpage}
\usepackage{amsmath}
\usepackage{cite}

\title{Kelvin-Helmoltz Instability}
\author{Jonathan Elsner}
\date{December 11\textsuperscript{th}, 2024}

\begin{document}

\maketitle

\section{Abstract}

% Maybe look at MIT guide to abstract writing?
Kelvin-Helmholtz (KH) instability is a process that occurs at the interface
between two fluids of two differing densities, creating vortex waves. KH
Instability is a commonly occurring process in nature: it is one of the driving
forces of mixing in the ocean %cite! Georgy isn't good enough
and can sometimes be seen in cloud formations. %picture?

The key characteristic of KH instability is the distance between the waves at
the interface. This changes over time and is dependent on both the difference in
density between the fluids and the flow rate of each fluid.

We follow a procedure outlined in \cite{kh-instability-demo} to recreate this
phenomenon in the lab to analyze the results and determine the relationship
between fluid speed, fluid density, and instability wavelength. We also analyze
the evolution of the waves over time.
% Talk about findings?

\section{Procedure}

To carry out the experiment, we followed the following rough procedure:

\begin{enumerate}
    \item Fill tank halfway with fresh water
    \item Mix salt for desired salinity. We calculated how much salt we wanted
    for a desired concentration, mixed that together, then verified the salinity
    with a handheld refractometer.
    \item Slowly fill the tank to full with the denser salt water.
    \item Return tank to level and allow the interface to stabilize
    \item Swiftly tilt tank to develop the instability.
\end{enumerate}

\section{Results and Analysis}

\section{Conclusion}
% Do I need this?

\bibliography{bibliography}{}
\bibliographystyle{plain}

\end{document}