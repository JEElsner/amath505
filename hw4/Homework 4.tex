\documentclass{article}

\usepackage{fullpage}
\usepackage{amsmath}
\usepackage{amssymb}
\usepackage[thinc]{esdiff}
\usepackage{hyperref}

\title{Homework 4}
\author{Jonathan Elsner}
\date{October 10th, 2024}

\begin{document}

\maketitle

\section{The Bathtub Vortex}

\subsection{}
From the internet\footnote{\url{https://www.me.psu.edu/cimbala/me320/Lesson_Notes/Fluid_Mechanics_Lesson_11C.pdf}},
the \(r\) component of the polar form for Navier-Stokes is:
\[\rho \left(\frac{\partial u_r}{\partial t} + u_r \frac{\partial u_r}{\partial r} + u_\theta \frac{\partial u_\theta}{\partial \theta} + u_z \diffp{u_r}{z} - \frac{u_\theta^2}{r} \right) = -\diffp{P}{r} + \rho g_r + \mu \left(\nabla^2 u_r - \frac{u_r}{r^2} - \frac{2}{r^2} \diffp{u_\theta}{\theta}\right)\]
Taking the assumptions that: \(\partial/\partial t = 0\), \(u_\theta \gg u_r, u_z\), rotational-axis symmetry, and \(P \propto r\) only, many simplifications occurr:
\[\rho \left(\frac{- u_\theta^2}{r} = - \frac{\partial P}{\partial r}\right)\]
Similarly, for the \(z\) component, we have:
\[\rho \left(\diffp{u_z}{t} + u_r \diffp{u_z}{r} + \frac{u_\theta}{r} \diffp{u_z}{\theta} + u_z \diffp{u_z}{z}\right) =  - \diffp{P}{z} + \rho g + \mu \nabla^2 u_z\]
Which under similary assumptions simplifies to:
\[0 = - \diffp{P}{z} + \rho g\]
and we get the good 'ol
\[P = \rho gz \]

\subsection{}

The energy conservation law for a surface parcel, assuming velocity is wholly azimuthal, is:
\[\frac{1}{2} \rho u_\theta^2 + \rho g z = C\]

\subsection{}
First, using the fact that angular momentum is \(L = \rho u_\theta r\), solving for \(u_\theta\), we get that
\[\frac{-L^2}{\rho r^3} = - \diffp{P}{r} \]
solving,
\begin{align*}
\int_r^\infty \frac{-L^2}{\rho r^3} dr &= - \int_r^\infty \diffp{P}{r} dr \\
\frac{L^2}{2 \rho r^2} &= P_\infty - P(r) \\
P(r) &= \rho g h_\infty - \frac{L^2}{2 \rho r^2}
\end{align*}
Where \(P_\infty\) is the pressure far from the vortex (i.e. atmospheric
pressure), and \(h_\infty\) is the height of the water far from the
vortex. Combining this with the derived hydrostatic bounds,
\begin{align*}
    \rho g h_\infty - \frac{L^2}{2 \rho r^2} &= \rho g z \\
    h_\infty - \frac{L^2}{2 \rho^2 g r^2} &= z
\end{align*}
Setting \(H(r) = z\),
\[ H(r) = h_\infty - \frac{L^2}{2 \rho^2 g r^2}\]
Returning to Bernoulli's law and noting that \(u_\theta = V(r)\):
\begin{align*}
    \frac{1}{2}\rho \left[V(r)\right]^2 + \rho g H(r) &= C/\rho \\
    \left[V(r)\right]^2 + 2 g H(r) &= C/\rho \\
    V(r) &= \pm \sqrt{C/\rho - 2g H(r)}
\end{align*}
We have an unknown constant \(C\). This represents the total energy in the system, both from potential energy in the height, and kinetic energy in the radial motion.

\subsection{}
Yes. It has to be. I refuse to provide more evidence. Mark me down I dare you.

\subsection{}
Since \(\nabla^2 \vec u = - \text{curl}\ \omega\), and \(u_\theta = L/\rho r\):
\begin{align*}
\nabla^2 u_\theta &= \frac{1}{r} \diffp{}{r} \left(r \diffp{u_\theta}{r}\right) \\
&= \frac{1}{r} \diffp{}{r}\left(r \frac{L}{\rho r}\right) \\
&= \frac{1}{r} \diffp{}{r}\left(\frac{L}{\rho}\right) \\
&= 0
\end{align*}
Implying \(\omega_z = 0\), which we expect, since the particles far from the vortex center are only revolving around the vortex axis, but not rotating themselves.

\section{Boundary Layer Thickness in the Bathtub Vortex}

\subsection{}
Now, assuming the velocity is almost purely radial in the boundary layer,
\[\rho \left(\diffp{u_r}{t} + u_r \diffp{u_r}{r} + \frac{u_\theta}{r} \diffp{u_r}{\theta} + u_z \diffp{u_r}{z} - \frac{u_\theta^2}{r}\right) = -\diffp{P}{r} + \rho g_r  + \mu\left(\nabla^2 u_r - \frac{u_r}{r^2} + \frac{2}{r^2}\diffp{u_r}{\theta}\right)\]
With our simplifying assumptions, this becomes:
\[\rho u_r \diffp{u_r}{r} = - \diffp{P}{r} + \mu \left(\nabla^2 u_r - \frac{u_r}{r^2}\right)\]
\subsection{}
I have literally no idea what to do after that.
\subsection{}
\subsection{}
\subsection{}

\end{document}